%%%%%%%%%%%%%%%%% DO NOT CHANGE HERE %%%%%%%%%%%%%%%%%%%% 
%%%%%%%%%%%%%%%%%%%%%%%%%%%%%%%%%%%%%%%%%%%%%%%%%%%%%%%%%%{
    \documentclass[twoside,11pt]{article}
    %%%%% PACKAGES %%%%%%
    \usepackage{pgm2016}
    \usepackage{amsmath}
    \usepackage{algorithm}
    \usepackage[noend]{algpseudocode}
    \usepackage{subcaption}
    \usepackage[english]{babel}	
    \usepackage{paralist}	
    \usepackage[lowtilde]{url}
    \usepackage{fixltx2e}
    \usepackage{listings}
    \usepackage{color}
    \usepackage{hyperref}
    \usepackage{listings}
    \usepackage{auto-pst-pdf}
    \usepackage{pst-all}
    \usepackage{pstricks-add}
    \hypersetup{
    colorlinks=true,
    linkcolor=blue,
    filecolor=magenta,      
    urlcolor=cyan,
    }
    
    %%%%% MACROS %%%%%%
    \algrenewcommand\Return{\State \algorithmicreturn{} }
    \algnewcommand{\LineComment}[1]{\State \(\triangleright\) #1}
    \renewcommand{\thesubfigure}{\roman{subfigure}}
    \definecolor{codegreen}{rgb}{0,0.6,0}
    \definecolor{codegray}{rgb}{0.5,0.5,0.5}
    \definecolor{codepurple}{rgb}{0.58,0,0.82}
    \definecolor{backcolour}{rgb}{0.95,0.95,0.92}
    \lstdefinestyle{mystyle}{
       backgroundcolor=\color{backcolour},  
       commentstyle=\color{codegreen},
       keywordstyle=\color{magenta},
       numberstyle=\tiny\color{codegray},
       stringstyle=\color{codepurple},
       basicstyle=\footnotesize,
       breakatwhitespace=false,        
       breaklines=true,                
       captionpos=b,                    
       keepspaces=true,                
       numbers=left,                    
       numbersep=5pt,                  
       showspaces=false,                
       showstringspaces=false,
       showtabs=false,                  
       tabsize=2
    }
    \lstset{style=mystyle}
%%%%%%%%%%%%%%%%%%%%%%%%%%%%%%%%%%%%%%%%%%%%%%%%%%%%%%%%%% 
%%%%%%%%%%%%%%%%%%%%%%%%%%%%%%%%%%%%%%%%%%%%%%%%%%%%%%%%%% }

%%%%%%%%%%%%%%%%%%%%%%%% CHANGE HERE %%%%%%%%%%%%%%%%%%%% 
%%%%%%%%%%%%%%%%%%%%%%%%%%%%%%%%%%%%%%%%%%%%%%%%%%%%%%%%%% {
\newcommand\course{COMP201}
\newcommand\courseName{Lab session}
\newcommand\semester{Fall 2020}
\newcommand\assignmentNumber{2}
\newcommand\assignmentDate{10/11/2020}

%%%%%%%%%%%%%%%%%%%%%%%%%%%%%%%%%%%%%%%%%%%%%%%%%%%%%%%%%%
    \ShortHeadings{Koc University -  \course }{}
    \firstpageno{1}
    \begin{document}
    \title{Lab Exercise \assignmentNumber}
    \maketitle
%%%%%%%%%%%%%%%%%%%%%%%%%%%%%%%%%%%%%%%%%%%%%%%%%%%%%%%%%%
In order to better understand bit representations of integers and floating-points, following exercise is designed. It is required to perform this exercise and submit your work to Blackboard by the end of the lab session. You are required to perform all operations using vi text editor.

\section{Materials}
You can reach the materials both from the Blackboard and the following git repository:

\begin{lstlisting}[language=bash] 
        $ git clone https://github.com/fnegahbani19/COMP201-Lab-B-02.git\end{lstlisting}

Note: If you are facing public key error on your local machine, you need to generate and add a public key to your ssh agent. You can find instructions \href{https://docs.github.com/en/enterprise/2.15/user/articles/generating-a-new-ssh-key-and-adding-it-to-the-ssh-agent}{here}. If you want to set this later you can also open \href{https://github.com/fnegahbani19/COMP201-Lab-B-02}{this page} and download the repository as a .zip file.

\section{Instructions}

In this task, you are expected to implement two functions that perform bit-wise operations and 
determine the bit representations of two floating point numbers. The functions that you have to 
implement are as follows.

\begin{itemize}
\item int getByte(int input, int byte\_position)

This function accepts two integers as inputs. It returns the byte in the integer input at the location pointed by byte\_position. The byte is numbered from 0 (least significant byte) to 3 (most significant byte). For example, getByte(0x12345678,1) = 0x56 because the byte at position 1 is 0x56.

\item int bitSum16bit(int input)

This function accepts a 16-bit unsigned number as an input, and counts the number of 1-bits in 
that number. For example, bitSum16bit(7) = 3 and bitSum16bit(17) = 2.

\end{itemize}

Please note that you are not allowed to use loop statements in any of these functions. 

In addition to writing these functions, you also have to determine the bit representations of 
224 and -0.375 based on the 8-bit floating representation in the page 6 of lecture 6 slides, 
and write the values of the bit representations as 1-byte hexadecimal numbers. In this 8-bit 
representation, 1 bit is allocated for sign, 4 bits are for exponent, and 3 bits are for fraction.

\section{Compilation and Correctness Checking}

After finishing your implementation,you need to compile and check the correctness of your code. To compile the code, run `make install`, and to check the correctness of the code, run `make test`.

\section{Submission}

You are expected to upload your project as a compressed file to the Blackboard whether you are working on your local machine or you are using "repl.it".


\newpage

\end{document}
